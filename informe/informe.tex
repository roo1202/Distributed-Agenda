\documentclass[11pt,a4paper]{article}

\usepackage[T1]{fontenc}
\usepackage[utf8]{inputenc} % Asegúrate que tu archivo .tex está en UTF-8
\usepackage[spanish]{babel}
\usepackage{amsmath}

\title{Informe}
\author{
  Roger Fuentes Rodr\'iguez \\
  Kevin Manzano Rodr\'iguez
}
\date{}

\begin{document}

\maketitle

\tableofcontents

\section{Arquitectura}

\subsection{Organizaci\'on del sistema distribuido}
Implementaremos una arquitectura de microservicios, con los principales servicios: 
\begin{itemize}
    \item Autenticaci\'on (gesti\'on de usuarios, autenticaci\'on e identificaci\'on)
    \item Grupos (gesti\'on de los grupos de los usuarios, incluyendo roles y jerarquías)
    \item Agendas (gesti\'on de eventos y reuniones)
    \item Notificaciones (envío de notificaciones en tiempo real a los usuarios)
    \item Base de datos distribuida (almacena la informaci\'on)
\end{itemize}

\subsection{Distribuci\'on de servicios en ambas redes Docker}
Tendremos dos redes de Docker separadas: 
\begin{itemize}
    \item \textbf{Cliente (frontend)}
    \item \textbf{Servidor (backend y base de datos)}
\end{itemize}
El servidor alberga los servicios anteriormente mencionados.

\section{Procesos}

\subsection{Tipos}
\begin{enumerate}
    \item Autenticaci\'on
    \begin{enumerate}
        \item Gesti\'on de usuarios
        \item Autenticaci\'on y autorizaci\'on
    \end{enumerate}
    \item Grupos
    \begin{enumerate}
        \item Gesti\'on de grupos
    \end{enumerate}
    \item Agenda
    \begin{enumerate}
        \item Gesti\'on de agenda y reuniones
    \end{enumerate}
    \item Notificaciones
    \begin{enumerate}
        \item Envío de notificaciones en tiempo real
    \end{enumerate}
    \item Base de datos
    \begin{enumerate}
        \item Gesti\'on de datos
    \end{enumerate}
\end{enumerate}

\subsection{Organizaci\'on en una instancia}
Los procesos estar\'an en instancias separadas para cada microservicio, cada servicio tiene su propia instancia de contenedor Docker.

\subsection{Tipo de patr\'on de dise\~no}
El sistema utilizar\'a peticiones as\'incronas (\texttt{async}) para las llamadas HTTP. Para optimizar el envío de notificaciones, se emplear\'an hilos que permitan paralelizar ese proceso.

\section{Comunicaci\'on}
Tendremos comunicaci\'on cliente-servidor a trav\'es de APIs RESTful. Entre servidores utilizaremos \texttt{ZMQ} y, entre procesos, \texttt{RPC}.

\section{Coordinaci\'on}
Para la coordinaci\'on entre servidores utilizaremos \texttt{PUB-SUB} que nos brinda \texttt{ZMQ}, y para el acceso exclusivo a recursos emplearemos \texttt{locks}. Con respecto a la toma de decisiones, implementaremos un algoritmo de consenso de quórum.

\section{Nombrado y localizaci\'on}
Utilizaremos \texttt{CHORD} como se vio en clases. Tendremos un anillo de tamaño $2^m$ y asignaremos los recursos a una clave utilizando la funci\'on de \texttt{hash} \texttt{SHA-256}. Para acceder a un recurso, emplearemos las \textit{finger tables} almacenadas en los nodos.

\section{Consistencia y replicaci\'on}
Dado que utilizaremos \texttt{CHORD}, para garantizar nivel 2 de tolerancia a fallos, implementaremos replicaci\'on m\'ultiple (\(K\)-réplicas) para distribuir las copias de los datos entre varios nodos (por ejemplo, consecutivos). Usaremos consistencia eventual combinada con reparaci\'on autom\'atica de r\'eplicas.

\section{Tolerancia a fallas}
Utilizaremos mecanismos de \textit{health checks} a los nodos vecinos de manera peri\'odica para verificar si est\'an activos. En caso de que un nodo falle, se asegura que las claves asignadas a ese nodo sean rápidamente reasignadas (gracias a la replicaci\'on). Para la recuperaci\'on de datos, las réplicas almacenadas en otros nodos sincronizan su informaci\'on una vez que el nodo se recupere.

\section{Seguridad}
Durante el tránsito, utilizaremos \texttt{TLS} para cifrar la comunicaci\'on entre los nodos de \texttt{CHORD}. Los nodos deben verificar la autenticidad de otros nodos, por lo que implementaremos una autenticaci\'on basada en certificados. Para la autenticaci\'on de los usuarios, usaremos \texttt{JWT}.

\end{document}
